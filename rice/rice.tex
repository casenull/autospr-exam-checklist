\documentclass[a4paper,10pt]{article}

\title{Satz von Rice}
\author{}
\date{}

\begin{document}
\maketitle
Gegeben sei eine Sprache $THREE_{TM}$, welche aus Beschreibungen von Turing-Maschinen $\langle M_1 \rangle, \langle M_1 \rangle, \dots$ besteht, welche jeweils Sprachen mit maximal drei Wörtern akzeptieren.

\[
    THREE_{TM} = \{\langle M \rangle | M \textnormal{ ist eine TM}, |L(M)| \leq 3\}
\]

Es soll gezeigt werden, dass $THREE_{TM}$ nicht entscheidbar ist. Dies kann gemacht werden, wenn die folgenden beiden Eigenschaften erfüllt sind:
\begin{enumerate}
    \item Es handelt sich bei der Eigenschaft, dass maximal drei Wörter akzeptiert werden, um eine \textbf{Spracheigenschaft}.
    
    Dies ist bewiesen, wenn $L(M_1) = L(M_2)$ nur \emph{genau dann} gilt, wenn 
    \[
        \langle M_1 \rangle\in THREE_{TM} \land \langle M_2 \rangle \in THREE_{TM}
    \]
    oder
    \[
        \langle M_1 \rangle\notin THREE_{TM} \land \langle M_2 \rangle \notin THREE_{TM}
    \]

    \item Die Eigenschaft, dass maximal drei Wörter akzeptiert werden, ist \textbf{nichttrivial}.
    
    Dies ist bewiesen, wenn $L(M_1) \neq L(M_2)$ zutrifft, da:
    \[
        \langle M_1 \rangle \in THREE_{TM} \land \langle M_2 \rangle \notin THREE_{TM}
    \]
    oder
    \[
        \langle M_1 \rangle \notin THREE_{TM} \land \langle M_2 \rangle \in THREE_{TM}
    \]
\end{enumerate}
\end{document}